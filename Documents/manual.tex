% Every Latex document starts with a documentclass command
\documentclass[a4paper, 11pt]{article}

% Load some packages
\usepackage{graphicx} % This allows you to put figures in
\usepackage{natbib}   % This allows for relatively pain-free reference lists
\usepackage[left=2cm,top=3cm,right=2cm]{geometry} % The way I like the margins
\usepackage{framed}

\newcommand{\rjobject}{{\tt RJObject}}
\newcommand{\dnest}{{\tt DNest3}}

% This helps with figure placement
\renewcommand{\topfraction}{0.85}
\renewcommand{\textfraction}{0.1}
\parindent=0cm

% Set values so you can have a title
\title{RJObject Manual}
\author{Brendon J. Brewer}
\date{\today}

% Document starts here
\begin{document}

% Actually put the title in
\maketitle

\section{Introduction}
RJObject is a C++ library for implementing certain kinds of trans-dimensional
hierarchical models (basically `mixture models')in DNest3. I developed this
library after realising that many models I was implementing all had this
common structure.
Throughout this manual, I will assume you're proficient at
DNest3. If not, please read the manual for that project first.

\section{Dependencies}
As well as all the DNest3 dependencies, it's best if you add some of DNest3's
directories to your various PATH variables. Specifically, things will work best
if you put the location of the DNest3 libraries (the location of
the {\tt libdnest3.so} file) to your {\tt LIBRARY\_PATH} and
{\tt LD\_LIBRARY\_PATH}. Also add the directory containing the DNest3 header
files ({\tt *.h}) to your {\tt CPLUS\_INCLUDE\_PATH}. Finally, add the
directory containing DNest3's {\tt postprocess.py} and {\tt dnest\_plots.py} to
your {\tt PYTHON\_PATH}.

\section{Compiling}
If you have all the dependencies, and have all the paths set up correctly as
specified above, you should be able to compile RJObject by going into the
root RJObject directory and invoking {\tt make}. Alternatively you can follow
the {\tt cmake} instructions in the README.

\section{Specifying models}
When using RJObject, you're really using DNest3. So to implement a model, you
need to write a DNest3 model class and specify the fromPrior, perturb,
logLikelihood methods etc. All RJObject provides is a quick and relatively
painless way of incorporating a trans-dimensional object into your model and
specifying heirarchical priors for its parameters. For example, consider
the SineWave example from the RJObject paper. There are an unknown number of
sinusoids $N$, and each of the sinusoids has an unknown period $T_i$, amplitude
$A_i$, and phase $\phi_i$. To use DNest3 on this problem, you'd have to write
the fromPrior and perturb methods for all of these parameters, which is
potentially very annoying. You'd have to decide such things as how to propose
to change $N$, how frequently to propose a move for one or a few of the
amplitudes, periods, and phases, and the nature of those moves. This is even
more complicated when you want a heirarchical prior for the amplitudes, periods,
and phases.

RJObject provides sensible default moves, drastically reducing the amount of
work required to write the model class. In the Examples/SineWaves directory,
the model is specified in the MyModel class. If you open MyModel.h you'll
notice the MyModel class doesn't have very many data members:

\begin{framed}
\begin{verbatim}
class MyModel:public DNest3::Model
{
    private:
        RJObject<MyDistribution> objects;
        double sigma;

        std::vector<long double> mu;
}
\end{verbatim}
\end{framed}

The variable {\tt objects}, of type {\tt RJObject<MyDistribution>}, represents
$\left\{N, \{A_i, T_i, \phi_i\}_{i=1}^N\right\}$. The only other parameter
is the noise standard deviation $\sigma$. The vector of long doubles is
the ``mock signal'' used in the likelihood function.



\end{document}

